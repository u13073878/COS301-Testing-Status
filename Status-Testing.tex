\documentclass[a4paper,12pt]{article}
\usepackage[hidelinks]{hyperref}
\usepackage{graphicx}
\usepackage{float}
\usepackage{caption}
\begin{document}
\begin{center}

%Cover page
\Huge\textbf{Testing Report\\}
																											
\vspace{2 cm}

\LARGE\textbf{Group Name:} Testing Status\newline
 
\vspace{0.5 cm}
\begin{tabular}{lr}
Daniel Christopher Alves Ara\"{u}jo&13073878\\ 
Taariq Ghoord&10132806\\
Lerato Molokomme&11197961\\
Semaka Malapane&13081129\\
Mpedi Mello&11210754\\
Ryno Pierce&12003922\\
Lutfiyya Razak&10198408\\
Frederick Snyman&13028741\\
Keagan Thompson&13023782\\
\end{tabular}

\vspace{1cm}
\textbf{Git repository link:\\}
\url{https://github.com/u13073878/COS301-Testing-Status}

\vspace{1cm}
\textbf{Date:} 24 April 2015
\end{center}
\pagenumbering{gobble}
\newpage

%table of contents
\tableofcontents
\pagenumbering{arabic}
\newpage

%github links 
%Group A - http://github.com/lana01/Status-A
%Group B - https://github.com/BuzzSpaceB/Status

%----------------1. Assess Profile ---------------
\section{assessProfile Use case}
\subsection{group A - assessProfile}
%add test results for group A assessProfile here
%Mpedi


\subsection{group B - assessProfile}
%add test results for group B assessProfile here
%Lerato



\newpage
%----------------2. Set status calculator ---------------
\section{setStatusCalculator Use case}
\subsection{group A - setStatusCalculator}
%add test results for group A setStatusCalculator here
%Lutfiyya


\subsection{group B - setStatusCalculator}
%add test results for group B setStatusCalculator here
%Taariq



\newpage
%----------------3. Get status for profile ---------------
\section{getStatusForProfile Use case}
\subsection{group A - getStatusForProfile}
%add test results for group A getStatusForProfile here


\subsection{group B - getStatusForProfile}
%add test results for group B getStatusForProfile here
Status B provided a \textit{getStatusForProfile} as in Figure 41 of the master specifications. No pre- or post-conditions are tested for in the code written by Status B, however none were provided in the given master specifications.

The function that Status B provided takes as parameters the ID of the user being queried.

Nodeunit was used for doing unit tests on this code, and the code of the unit tests appears in the figure below:

	\begin{figure}[H]
		\centering
		\includegraphics[width=1.0\textwidth]{Figures/getStatusForProfile_unittests.png}
		\caption{Unit tests for getStatusForProfile}
	\end{figure}
	
Executing the tests above resulted in the following output:

	\begin{figure}[H]
		\centering
		\includegraphics[width=1.0\textwidth]{Figures/getStatusForProfile_results.png}
		\caption{Results of the unit tests for getStatusForProfile}
	\end{figure}
	
The tests that are run, checks the status value for six users present in the database, with differing status values. All six tests pass successfully and the function \textit{getStatusForProfile} appears to hav been implemented correctly as per the master specification. These test cases cover 100\% of the use cases as defined in the master specification.

\newpage
%----------------4. Create appraisal type ---------------
\section{createAppraisalType Use case}
\subsection{group A - createAppraisalType}
%add test results for group A createAppraisalType here
%Keagan

\subsection{group B - createAppraisalType}
%add test results for group B createAppraisalType here
%Semaka



\newpage
%----------------5. Activate appraisal type ---------------
\section{activateAppraisalType Use case}
\subsection{group A - activateAppraisalType}
%add test results for group A activateAppraisalType here
%Keagan and Ryno

\subsection{group B - activateAppraisalType}
%add test results for group B activateAppraisalType here
%Chris

Status B did not provide an explicit \textit{activateAppraisalType} function that matches Figure 45 of the project specifications.
However, Status B does provide two functions, \textit{activePeriod()} and \textit{setAppraisal()}.

The code for the unit tests is given by the figure below. For the test, NodeUnit was used.

	\begin{figure}[H]
		\centering
		\includegraphics[width=1.0\textwidth]{Figures/activateAppraisalTypeUnitTests.png}
		\caption{Unit test for activateAppraisalType}
	\end{figure}

The following was the output given by NodeUnit.

	\begin{figure}[H]
		\centering
		\includegraphics[width=0.8\textwidth]{Figures/activateAppraisalTypeUnitTestOutput.png}
		\caption{Output of unit tests for activateAppraisalType}
	\end{figure}

The first test, \textit{activePeriodTest}, passes. \textit{activePeriod()} takes two strings as parameters that represent the starting and ending date. However, it does no testing to ensure that the dates are in the correct format supported by JavaScript. Secondly, it does not have a callback parameter to specify a callback function, so it cannot be run asynchronously.

The section function, \textit{setAppraisal()}, fails. The function receives two parameters, a \textbf{post\_id} and and \textbf{appraisal\_name} to find a record in the database matching the \textbf{post\_id} and assign the \textbf{appraisal\_name} to \textbf{appraisal\_id} in the database. The function has an error that attempts to assign a null value to \textbf{post.appraisal\_id} in the database. 

The pre-conditions are not tested in either function, although a assumption can be made about a buzz space being active. However, \textit{setAppraisal()} does not test whether or not the active period is before the current date or time.

Due to the \textit{setAppraisal()} failing, the function does not meet the post-condition of appraisalTypeAssignment being persisted to the database.

\newpage
%----------------6 Assign appraisal to post ---------------
\section{assignAppraisalToPost Use case}
\subsection{group A - assignAppraisalToPost}
%add test results for group A assignAppraisalToPost here
%Keagan and Ryno

\subsection{group B - assignAppraisalToPost}
%add test results for group B assignAppraisalToPost here
%Frikkie



\end{document}